% !TEX root = ../master-thesis.tex






Single-atom imaging has become an essential tool in quantum simulation and many-body physics with ultracold atoms. It allows to directly observe individual atoms, providing detailed insights into quantum phenomena that previously could only be inferred indirectly from ensemble measurements. By visualizing individual atomic positions, single-atom imaging enables measurements of multi-point correlation functions, revealing subtle and complex correlations within strongly interacting quantum systems. This capability significantly enhances our understanding of quantum phases and dynamics.

\textbf{Quantum Gas Microscopy with Raman Sideband Cooling.} The foundation of modern single-atom imaging was laid with the development of quantum gas microscopes, which combine high-numerical-aperture optics with fluorescence detection to resolve individual atoms trapped in optical lattices. The breakthrough came with the first demonstrations by \cite{bakr_quantum_2009, sherson_single-atom-resolved_2010}, who achieved single-site resolution imaging of bosonic atoms in Hubbard-regime optical lattices.

A critical advancement was the implementation of Raman sideband cooling during the imaging process, first demonstrated by \cite{lester_raman_2014}. This technique addresses a challenge in fluorescence imaging: the heating of atoms due to photon recoil during the scattering process. Traditional fluorescence imaging required deep optical lattices to localize atoms, but even then, the random momentum kicks from scattered photons could cause atoms to hop between lattice sites or be lost from the trap entirely.

The Raman sideband cooling approach simultaneously cools and images the atoms by driving transitions that remove vibrational quanta while maintaining the atoms in a cycling transition. This was successfully extended to fermionic atoms by \cite{cheuk_quantum-gas_2015}.
% , who demonstrated single-atom imaging of $^{40}$K atoms with detection fidelities above 95\%. 
The technique proved particularly valuable for fermions, where the broader $D_2$ transitions and more complex hyperfine structure make conventional sub-Doppler cooling less effective.

Quantum gas microscopy with Raman sideband cooling offers ultra-high spatial resolution approaching the optical diffraction limit (approximately 500 nm) and high detection fidelity
% typically exceeding 95\% 
due to the large number of scattered photons per atom. The preservation of quantum coherence is maintained as atoms remain near the motional ground state during imaging, making the technique compatible with complex atomic species including those with unresolved hyperfine structure.

However, this approach comes with significant technical complexity, requiring multiple laser frequencies, precise polarization control, and deep lattice potentials. The imaging process is relatively slow with exposure times of 100 ms to 1 s, limiting experimental cycle rates. 
% Each atomic species requires specific optimization of cooling transitions and laser parameters, and the system remains sensitive to heating from imperfect beam alignment and residual light shifts.

\textbf{Spin-Resolved Detection in Quantum Gas Microscopy.} The extension to spin-resolved imaging has been crucial for studying magnetic phenomena in quantum simulators. The first demonstrations of spin-resolved quantum gas microscopy were achieved by \cite{boll_spin-_2016, parsons_site-resolved_2016} in 2016, enabling the direct observation of antiferromagnetic correlations in Fermi-Hubbard systems. Several distinct approaches have been developed to achieve spin discrimination in quantum gas microscopes, each with specific advantages and applications.

Sequential state-selective imaging sequences utilize the hyperfine structure of atoms to selectively image different spin states through time-separated exposures. In this approach, atoms in different spin states are imaged one after the other by using optical transitions that address one state while leaving the other largely unaffected. The method typically involves optically pumping or "blowing away" one spin component first, followed by imaging the remaining spin in a second exposure. For example, \cite{parsons_site-resolved_2016} demonstrated spin-resolved site occupancy by applying a resonant laser pulse that ejected atoms in one hyperfine state with a scattering rate approximately 1700 times higher than for the other state, removing over 99\% of the target-spin atoms while causing only 0.06\% loss in the opposite spin. Similarly, \cite{cheuk_quantum-gas_2015} employed a microwave transfer followed by a hyperfine-selective light pulse to remove one spin state with 98\% fidelity while causing only 0.3\% loss in the other state. More recently, \cite{bourgund_formation_2025} demonstrated sophisticated optical pumping techniques for spin-selective manipulation in mixed-dimensional Fermi-Hubbard systems, enabling the study of stripe formation in strongly correlated quantum matter.

Spatial separation techniques physically separate different spin states before imaging through various mechanisms. The Stern-Gerlach approach uses magnetic field gradients to deflect atoms of different spin states during time-of-flight expansion, creating spatially separated clouds that can be imaged individually. This classical method has been widely employed since early spinor Bose-Einstein condensate experiments, providing robust spin discrimination for bulk measurements, though typically without single-site resolution.

The bilayer approach represents a more sophisticated spatial separation technique using state-dependent lattice shifts to move different spin components to separate lattice planes. \cite{preiss_quantum_2015} pioneered this method in a two-plane lithium-6 lattice, where a magnetic field gradient enabled spin-dependent transport of one hyperfine state into a second layer. This technique was further refined by \cite{boll_spin-_2016}, who demonstrated simultaneous spin-and-charge resolved quantum gas microscopy using a state-dependent superlattice in one-dimensional Hubbard chains. Their approach used a local "spin address" pulse to separate the two spin states into different wells of each double-well pair, enabling fluorescent imaging of one spin state in the original layer and the other in the second layer.

Polarization-based detection exploits the polarization dependence of optical transitions to distinguish spin states through the analysis of emitted fluorescence. By using a polarizing beam splitter after the imaging objective, $\sigma^+$ and $\sigma^-$ polarized fluorescence can be directed to different regions of the detector, enabling single-shot spin discrimination. This approach leverages optical selection rules where atoms in different spin states emit light with distinct polarization signatures, allowing simultaneous detection of both spin components without temporal separation. While conceptually elegant and offering the potential for truly simultaneous dual-spin imaging, this method requires careful optimization of the imaging transition and magnetic quantization axis to achieve sufficient polarization contrast between the spin states.

These various approaches to spin-resolved detection have enabled unprecedented insights into quantum magnetic phenomena, from the direct observation of antiferromagnetic ordering and spin correlations to the study of exotic phases such as quantum spin liquids and topological states. The choice of method depends on the specific experimental requirements, with sequential imaging offering high fidelity at the cost of temporal resolution, spatial separation providing robust discrimination for bulk measurements, bilayer techniques enabling simultaneous high-resolution detection, and polarization-based methods promising single-shot capability for both spin components.

\textbf{Free-Space Spin-Resolved Single-Atom Imaging.} A conceptually different approach was introduced by \cite{bergschneider_spin-resolved_2018}, who demonstrated the first spin-resolved single-atom imaging in free space without any confining potentials during detection. This method represents a significant departure from quantum gas microscopy techniques, offering both simplicity and speed while maintaining the capability to distinguish between different internal atomic states.

In the free-space approach, atoms are released from their traps and illuminated with resonant light for a brief period of approximately 20 $\mu$s. Each atom scatters approximately 20 photons, which is sufficient for detection with high fidelity of 99.4\%. The key innovation is the short exposure time, which minimizes the spatial diffusion due to photon recoil. During the 20 $\mu$s imaging pulse, atoms undergo a random walk with a characteristic size of approximately 4 $\mu$m, determined by the balance between photon number and recoil-induced motion. This approach has been further refined by \cite{su_fast_2025}, who achieved 2.4 $\mu$s imaging times while maintaining 99.4\% fidelity.

The original implementation by \cite{bergschneider_spin-resolved_2018} achieved spin resolution through sequential imaging using the stretched state $\ket{3}$. In their protocol, atoms initially in state $\ket{1}$ were first transferred to $\ket{3}$ and imaged, followed by transfer of $\ket{2}$ atoms to $\ket{3}$ for a second imaging sequence. While this sequential approach successfully demonstrated spin discrimination, it imposed certain limitations including potential atom loss between imaging steps and increased cycle time due to the required state transfers and multiple exposures.

For applications involving optical lattice systems, it is important to note that the spatial resolution of free-space imaging (approximately 4 $\mu$m) is insufficient to resolve individual lattice sites, which typically have spacings of 500-1000 nm. Therefore, single-site resolved imaging of atoms in optical lattices using free-space techniques would require implementation of a matter-wave magnifier to enhance the effective spatial resolution before the imaging process.

In the present work, we demonstrate a significant advancement by implementing simultaneous spin-resolved detection using both stretched states $\ket{3}$ and $\ket{6}$. This represents the first parallel imaging of both spin components in $^6$Li, eliminating the sequential transfer and imaging steps required in previous implementations. By utilizing polarization-selective detection paths, atoms in states $\ket{3}$ and $\ket{6}$ can be imaged simultaneously, with their fluorescence spatially separated on the detector through polarizing beam splitters. This parallel approach offers several advantages: reduced total imaging time, elimination of atom loss between sequential exposures, and simplified experimental protocols.

Free-space imaging offers remarkable simplicity, requiring only resonant laser light and a camera without complex cooling schemes. The technique provides exceptional speed with imaging times orders of magnitude faster than quantum gas microscope approaches, and shows universal applicability as it can be easily adapted to different atomic species. The absence of confining potentials enables momentum-sensitive measurements such as time-of-flight studies and velocity distributions, while the minimal cycle overhead allows for high experimental repetition rates.

The primary limitations remain the spatial resolution constraint of approximately 4 $\mu$m determined by recoil diffusion, the destructive nature of the measurement as atoms are lost or significantly displaced, and density constraints requiring sufficient spacing between atoms to avoid overlap of their fluorescence signals. Additionally, the lack of confinement prevents repeated measurements on the same atomic ensemble. Despite these limitations, the combination of speed, simplicity, and spin resolution makes free-space imaging particularly attractive for experiments with optical tweezer arrays, where atoms are naturally well-separated, and for applications requiring high temporal resolution or rapid experimental cycling.

\textbf{Comparative Analysis.} The choice between quantum gas microscopy and free-space imaging depends critically on the experimental requirements and system geometry. Each approach offers distinct advantages that make them optimal for different classes of experiments and scientific questions.

Quantum gas microscopy excels when ultra-high spatial resolution is essential, particularly in optical lattice experiments where individual lattice sites must be resolved with separations of 500-1000 nm. The technique's ability to achieve near-diffraction-limited resolution while maintaining high detection fidelity makes it indispensable for studying strongly correlated phases where site-by-site information is crucial. The capability to measure local observables such as density-density correlations, magnetic structure factors, and multi-point correlation functions has enabled groundbreaking studies of quantum magnetism, including the direct observation of antiferromagnetic ordering in Fermi-Hubbard systems \cite{boll_spin-_2016, parsons_site-resolved_2016} and the recent detection of stripe formation in mixed-dimensional systems \cite{bourgund_formation_2025}.

The Raman sideband cooling approach ensures that atoms remain close to their motional ground state during imaging, preserving quantum coherence and enabling repeated measurements or post-selection procedures. This capability has proven particularly valuable for studying entanglement properties and implementing quantum error correction protocols. However, the technical complexity remains substantial, requiring precise control of multiple laser frequencies, polarizations, and magnetic fields, along with deep optical lattices to ensure adequate confinement during the imaging process.

Free-space imaging, conversely, is optimal for experiments requiring high temporal resolution or when studying systems with naturally large spatial separations, such as atoms in optical tweezer arrays. The technique's exceptional speed, with imaging times of 2.4-20 $\mu$s compared to 100 ms-1 s for quantum gas microscopy, enables experimental cycle rates that are orders of magnitude faster. This speed advantage is particularly valuable for statistical measurements requiring large numbers of experimental repetitions or for studying rapid dynamical processes.

The simplicity of free-space imaging, requiring only resonant laser light and detection optics without complex cooling schemes, makes it attractive for proof-of-principle experiments and as a diagnostic tool. The absence of confining potentials during imaging enables unique capabilities such as momentum-resolved measurements through time-of-flight techniques, which are impossible with trapped imaging methods.

For optical lattice systems, however, the spatial resolution limitation of free-space imaging (approximately 4 $\mu$m) renders it insufficient for direct single-site resolution without additional magnification techniques. Implementation of matter-wave magnifiers could potentially bridge this gap, enabling the combination of free-space imaging's speed advantages with lattice-scale spatial resolution, though such schemes remain to be fully demonstrated experimentally.

The development of spin-resolved capabilities has been transformative for both approaches. While quantum gas microscopy has achieved sophisticated spin detection through bilayer techniques, state-selective sequences, and polarization-based methods, the recent advancement to simultaneous parallel imaging of both spin components in $^6$Li using states $\ket{3}$ and $\ket{6}$ represents a significant improvement over sequential methods. This parallel detection eliminates inter-exposure atom loss and reduces cycle times while maintaining the speed advantages inherent to free-space approaches.

Recent technological developments continue to push the boundaries of both methods. Fast imaging techniques \cite{su_fast_2025} have reduced quantum gas microscope exposure times to microseconds while maintaining sub-wavelength resolution, narrowing the speed gap between the approaches. Simultaneously, improvements in camera technology and laser stability are enhancing the fidelity and applicability of free-space methods across different atomic species and experimental geometries.

The choice between techniques often depends on the specific balance between spatial and temporal resolution requirements. For experiments studying equilibrium properties of strongly correlated systems in optical lattices, quantum gas microscopy remains unmatched in its ability to provide detailed spatial information. For dynamic studies, rapid diagnostics, or experiments with naturally separated atoms such as in tweezer arrays, free-space imaging offers compelling advantages in speed and simplicity.

Looking forward, both approaches remain active areas of development, with ongoing efforts to improve fidelity, reduce exposure times, and extend capabilities to new atomic species and experimental configurations. The continued evolution of these techniques promises to further expand the frontier of quantum simulation and many-body physics with ultracold atoms, enabling new classes of experiments that leverage the unique strengths of each imaging modality.

