% !TEX root = ../master-thesis.tex

\textbf{Frequency to position mapping.}
To extract the local intensities $P_{ij}$ from camera images, we need to determine which pixels correspond to which tweezer sites. For this purpose, we define an affine transformation from the drive frequency space $(\omega_{\mathrm{hor}}, \omega_{\mathrm{ver}})$ to image plane coordinates $(x, y)$:
\begin{equation*}
    \vc{r} = H \vc{\omega},
    \hspace{5 mm} \Leftrightarrow \hspace{5 mm} 
    \begin{pmatrix}
        x \\ y
    \end{pmatrix} = \begin{pmatrix}
        h_{11} & h_{12} & h_{13} \\
        h_{21} & h_{22} & h_{23}
    \end{pmatrix} 
    \begin{pmatrix}
        \sub{\omega}{hor} \\
        \sub{\omega}{ver} \\
        1
    \end{pmatrix}.
\end{equation*}
Here, $H$ is a $2 \times 3$ matrix calibrated from a set of measured spot positions. For example, one can measure $\vc{r}_j$ for random frequency vectors $\vc{\omega}_j \in [\omega_{\mathrm{min}},\, \omega_{\mathrm{max}}]$, construct the matrices $\omega_{ij}$ with $i \in \{\mathrm{hor}, \mathrm{ver}\}$ and $r_{ij}$ with $i \in \{x, y\}$, and solve the least-squares problem:
\begin{equation}
    r = H \omega,
    \hspace{0.5cm} \Rightarrow \hspace{0.5cm}
    r \omega^\mathrm{T} = H \omega \omega^\mathrm{T}
    \hspace{0.5cm} \Rightarrow \hspace{0.5cm}
    r \omega^\mathrm{T} \left(\omega \omega^\mathrm{T}\right)^{-1} = H.
    \label{eq:linreg-freq2pos}
\end{equation}

This transformation defines a region of interest around each tweezer, within which we compute the integrated pixel intensity after background subtraction. The resulting values are proportional to the optical powers $P_{ij}$.

\textbf{Linear reconstruction.}
The mapping from input amplitudes $\vc{a}$ to optical power is approximated by Eq.~\eqref{eq:taylerexp}. In the regime $a_i \in [0.85, 0.95]$, a linear approximation is sufficient\footnote{
    For wider amplitude ranges, higher-order terms can be added to the model. However, this is unnecessary in the present context.
}. We construct the Jacobian matrix $F'_{ji}$ by fitting a linear regression model to a dataset of amplitude–intensity pairs. The resulting crosstalk matrix is shown in Fig.~\ref{fig:control}b. It is approximately diagonal, with comparable diagonal entries and off-diagonal elements typically reaching up to 30\% in magnitude relative to the diagonal, due to power redistribution between neighboring tones. Crosstalk between the horizontal and vertical AODs remains negligible.

The quality of the linear fit for the $4 \times 4$ array is illustrated in Fig.~\ref{fig:control}e. The total intensity (Fig.~\ref{fig:control}d) scales linearly with the average input amplitude, yielding $R^2 > 0.99$. Relative residuals are normally distributed with width $0.3\%$, confirming the applicability of the model in this range.

\textbf{Power-aware optimization.}
\grey{In the presence of limited laser power and finite AOM diffraction efficiency, we prefer solutions where all amplitudes remain close to 1. This preference can be incorporated into the optimization objective. In addition to minimizing intensity imbalance, we penalize deviations of the average amplitudes from a target value (e.g., 0.9).}
