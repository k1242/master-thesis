% !TEX root = ../master-thesis.tex

\textbf{Solid-state limitations.} Despite decades of intensive research, fundamental questions about high-temperature superconductivity and many-body localization remain unresolved due to inherent limitations of solid-state systems. Real materials possess complex crystal structures with multiple orbital degrees of freedom, phonon interactions, and intrinsic disorder that obscure the underlying electronic physics \cite{koepsell_quantum_2021}. The limited tunability of material parameters requires synthesizing new compounds for each point in parameter space, making systematic studies of phase diagrams challenging. Furthermore, many key observables remain experimentally inaccessible in solid-state systems. Higher-order spin correlations, real-space magnetic textures, and single-site resolved quantities cannot be directly measured using conventional condensed matter probes, which typically access momentum-space or bulk averaged properties.

\textbf{Computational intractability.} Classical numerical approaches face fundamental obstacles when simulating strongly correlated fermionic systems. The dimension of the Hilbert space grows exponentially with system size, rendering exact diagonalization feasible only for very small clusters. Quantum Monte Carlo methods, while successful for certain bosonic systems, suffer from the fermion sign problem in the presence of frustration or doping, leading to exponentially growing statistical errors \cite{koepsell_quantum_2021}. Approximate methods such as density matrix renormalization group (DMRG) work well in one dimension but become inefficient for two-dimensional systems due to area-law violations in entanglement entropy. These computational limitations severely constrain theoretical understanding of the parameter regimes most relevant to high-temperature superconductivity and many-body localization.

\textbf{Key observables needed.} Progress in understanding both equilibrium and dynamical properties of the Fermi-Hubbard model requires access to observables that are difficult or impossible to measure in conventional systems. For high-temperature superconductivity, key quantities include real-space spin-charge correlations that reveal magnetic polaron formation, site-resolved doping profiles, and the evolution of magnetic correlations with charge carrier density \cite{koepsell_quantum_2021}. For dynamical phase studies, essential observables include local particle densities $\langle n_i(t) \rangle$, magnetization profiles $\langle \sigma^z_i(t) \rangle$, and multi-point correlations that distinguish thermal, Anderson localized, and many-body localized phases. Additionally, investigating dynamical phases requires the ability to prepare arbitrary initial states, control disorder realizations, and perform extensive statistical averaging over many experimental repetitions to extract meaningful signals from noisy quantum dynamics.