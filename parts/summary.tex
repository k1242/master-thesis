% !TEX root = ../master-thesis.tex

This thesis presented the development of experimental and numerical tools for future studies of the two-dimensional Fermi-Hubbard model with ultracold $^6$Li atoms.

The first contribution was the implementation of a spin-resolved free-space imaging system (Sec.~\ref{sec:imaging}). The system uses polarization-selective optics to simultaneously detect atoms in stretched states $|3\rangle$ and $|6\rangle$ with 99\% spin discrimination fidelity and 20 µs imaging time. This approach builds on established free-space imaging techniques \cite{bergschneider_spin-resolved_2018,su_fast_2025} but extends them to simultaneous two-state detection. The complete optical setup, distribution board, and image analysis software were developed as part of this work.

The second contribution involved developing control methods for two-dimensional optical tweezer arrays generated by crossed acousto-optic deflectors (Sec.~\ref{sec:tweezer}). Calibration protocols were implemented that reduce tweezer depth variations to 1\% across arrays through single-value feedback optimization. The spilling technique for deterministic atom number preparation, previously demonstrated in single tweezers \cite{serwane_deterministic_2011,holten_observation_2021} and two-dimensional arrays \cite{spar_programmable_2024}, was implemented in our system achieving 90\% fidelity for doublon preparation. While \cite{spar_programmable_2024} established crosstalk matrix characterization and single-value feedback for tweezer arrays, this work extends the approach to include higher-order nonlinear contributions and implements SVD-based decomposition during single value feedback to separately characterize and control contributions from horizontal and vertical AODs in crossed geometries. This enables simultaneous balancing of both AODs. Spin-selective spilling was demonstrated for specific removal patterns. However, full arbitrary pattern preparation requires an additional microwave antenna for $|1\rangle \leftrightarrow |2\rangle$ transitions, which is currently being in progress.

The third contribution was numerical simulation software for Fermi-Hubbard dynamics (Sec.~\ref{sec:fhmodel}). The GPU-accelerated implementation combines exact diagonalization for systems up to $10^4$ Hilbert space dimensions with Krylov methods for larger systems up to $10^9$ dimensions. Simulations of a 4-particle system in 16 sites demonstrated distinguishable signatures of thermal, Anderson localized, and many-body localized phases through density imbalance and entanglement entropy evolution.

The transfer to optical lattices for Hamiltonian evolution and subsequent dynamics measurements remain future objectives. The numerical simulations provide benchmarks for these planned experiments but have not yet been experimentally validated. The developed platform provides the technical foundation for future quantum simulation experiments. The combination of deterministic state preparation capabilities and spin-resolved detection, once integrated with the planned optical lattice system, will enable studies of strongly correlated phases in two-dimensional fermionic systems.


