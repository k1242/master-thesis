% !TEX root = ../master-thesis.tex

This chapter has detailed the implementation of a spin-resolved free-space imaging system for ${}^6$Li atoms in an optical tweezer array. The approach is tailored to fast-cycle experiments and leverages the intrinsic spacing of the array to bypass the need for sub-micron spatial resolution.

The free-space fluorescence method was motivated both conceptually and practically, with an emphasis on momentum-space dynamics during imaging. Theoretical considerations based on the SSH model support the use of alternating beam sequences to reduce spatial diffusion, thereby improving localization fidelity.

The optical setup includes two independent laser beams driving stretched-state cycling transitions, with modulation handled via AOMs and synchronized control electronics. A distribution board was constructed to combine and route the beams, ensuring symmetric delivery to the atoms.

The image analysis pipeline applies binarization and spatial filtering to extract single-atom signals from low-flux data. The use of fixed array geometry enables simplified region-of-interest (ROI) analysis and improves classification reliability. All components of the processing chain are implemented in a parallelized, vectorized framework to support high-throughput acquisition.

Spin information is extracted by mapping fluorescence from $\sigma^+$ and $\sigma^-$ transitions to distinct regions on the camera. A per-site classification algorithm assigns spin states based on signal strengths in these regions. Validation against calibrated single-atom counted experiments yields a classification accuracy of approximately 99\% under typical experimental conditions.

These techniques constitute a fast and reliable imaging solution suitable for experiments requiring spin-resolved readout, and are readily extensible to future array sizes and imaging geometries.
