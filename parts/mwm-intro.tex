% !TEX root = ../master-thesis.tex

\textbf{Motivation and overview.} 
Achieving single-site resolution in quantum gas microscopy is crucial for experiments probing quantum many-body phenomena. Conventional imaging techniques typically rely on high numerical aperture objectives to directly resolve atomic distributions in optical lattices. However, even with advanced optics, direct imaging is often limited by the fundamental diffraction limit and the intrinsic diffusion of atoms during photon scattering. In particular, for lattice spacings on the order of a micrometer or below—such as the bow-tie lattice spacing of 0.75$\mu$m used in our experiment—direct imaging becomes impractical due to both optical resolution limits and momentum kicks received by atoms from resonant imaging light, causing significant spatial diffusion.

To overcome these limitations, we utilize the technique of matter-wave magnification (MWM), which enables significant enlargement of the spatial distribution of atoms prior to the imaging process \cite{huang_construction_2024}. MWM operates by carefully engineering the evolution of atomic ensembles through controlled harmonic potentials. This method enlarges spatial patterns coherently, thus facilitating the detection of atomic distributions with improved resolution.

The principle underlying MWM can be understood from the perspective of phase-space dynamics. Consider atoms initially confined near the turning points of a harmonic potential with frequency $\omega_1$. After evolving for a quarter of the harmonic period $T_1/4=\pi/2\omega_1$, their spatial distribution transforms entirely into momentum space. Subsequently, atoms are subjected to a second, weaker harmonic potential characterized by frequency $\omega_2$ for another quarter period $T_2/4=\pi/2\omega_2$. During this second stage, the momentum-space distribution converts back into real space, resulting in magnification by a factor given by the ratio $\omega_1/\omega_2$ \cite{huang_construction_2024}. From the viewpoint of the Schrödinger equation, each evolution step corresponding to a quarter period in a harmonic potential functions as a Fourier transformation, converting between position and momentum space representations.


\begin{figure}
    \centering
    \addletter{130}{a}
    \includegraphics{fig-ai/mwm-scheme.pdf}
    \addletter{130}{b}
    \includegraphics{fig-py/mwm.pdf}
    \caption[Matter-wave magnification scheme and ROI adaptation]{
    \textbf{Matter-wave magnification scheme and ROI adaptation.}
    (a) Conceptual illustration of matter-wave magnification: atoms initially localized at the slope of a harmonic potential with frequency $\omega_1$ roll towards the center during a quarter period. Subsequent expansion in a shallower harmonic potential with frequency $\omega_2$ increases their spatial separation by the factor $\omega_1/\omega_2$.
    (b) Initial atom distribution (left) and magnified pattern after propagation (right), demonstrating distortion due to anharmonicities. Voronoi diagrams adapt the regions of interest to these distorted patterns, potentially improving atom detection fidelity.
    }
    \label{fig:mwm}
\end{figure}

However, practical implementation of MWM faces several challenges, notably potential anharmonicities that cause distortions of the magnified atomic pattern. These distortions necessitate adaptive methods for defining regions of interest (ROI) to ensure accurate atom detection. In this work, the application of Voronoi diagrams was introduced to adjust ROI based on actual atomic distributions post-magnification, thus potentially enhancing detection fidelity.
