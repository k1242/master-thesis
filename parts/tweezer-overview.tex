% !TEX root = ../master-thesis.tex


Quantum simulation with ultracold atoms requires precise control over initial many-body states to access regimes beyond thermal equilibrium and study specific quantum phenomena. Traditional loading methods from magneto-optical traps or degenerate gases produce stochastic particle distributions that limit the ability to prepare designer initial states necessary for studying complex quantum phases, far-from-equilibrium dynamics, or engineered many-body Hamiltonians. The development of deterministic loading and manipulation techniques has become essential for advancing quantum simulation capabilities, particularly in fermionic systems where spin degrees of freedom play crucial roles.

\textbf{Traditional Loading Methods and Limitations.} 
Conventional approaches to loading optical lattices rely on transferring atoms from magneto-optical traps (MOTs) or pre-cooled atomic clouds into the periodic potential created by interfering laser beams. In these methods, atoms are first captured and cooled in a MOT, then typically transferred to a magnetic or optical dipole trap for further evaporative cooling to quantum degeneracy. The resulting degenerate gas is subsequently loaded into the optical lattice by adiabatically ramping up the lattice depth while the harmonic confinement is maintained or modified.

This standard loading procedure suffers from fundamental limitations arising from its stochastic nature. When atoms are loaded from a thermal or quantum degenerate cloud, the number of particles at each lattice site follows Poissonian statistics, determined by the local chemical potential and temperature. For a uniform system at chemical potential $\mu$, the average site occupation is given by the local density, but fluctuations are unavoidable. Experimental demonstrations of Mott insulators in various atomic species have confirmed these statistical limitations. Early experiments with $^{87}$Rb achieved Mott insulator phases with site occupancy fidelities limited to approximately \red{60-70\%} for unit filling, primarily due to thermal fluctuations and imperfect adiabatic loading. Similar limitations were observed in fermionic $^{40}$K and $^6$Li systems, where achieving high-fidelity band insulators required very low temperatures and careful control of interaction strengths through Feshbach resonances.
% In the Mott insulator regime at unit filling, where on average one atom occupies each site, the probability of finding exactly $n$ atoms at a given site follows $P(n) = e^{-\langle n \rangle}\langle n \rangle^n/n!$ where $\langle n \rangle \approx 1$.
% This results in approximately 37\% empty sites, 37\% singly occupied sites, 18\% doubly occupied sites, and 8\% higher occupancies even under optimal conditions.


The harmonic confinement typically present in these experiments further complicates state preparation by creating spatial inhomogeneity. The varying local chemical potential across the trap leads to the formation of concentric "wedding cake" structures with different filling factors in different regions, making it difficult to achieve uniform occupancy over extended areas. While techniques such as digital micromirror device (DMD) compensation can flatten the trapping potential, the fundamental Poissonian loading statistics remain unchanged.

For spin-selective loading, traditional methods face additional challenges. Starting from a thermal cloud or degenerate gas, both spin components are typically loaded simultaneously with correlated fluctuations. Achieving specific spin patterns or controlled spin imbalances requires post-loading manipulation through techniques such as selective removal or magnetic field gradients, which can introduce heating and reduce overall atom numbers.

These limitations have motivated the development of alternative approaches that can circumvent the statistical nature of traditional loading and provide deterministic control over both spatial and spin degrees of freedom in lattice systems.


\textbf{SLM and DMD for Site-Selective Manipulation.}
The limitations of stochastic loading motivated the development of post-loading manipulation techniques using spatial light modulators (SLMs) and digital micromirror devices (DMDs) to achieve precise control over atomic distributions in optical lattices. These approaches enable site-selective addressing and manipulation after initial loading, allowing experimenters to "sculpt" desired atomic configurations from the initial Poissonian distribution.

Digital micromirror devices offer fast switching times and straightforward implementation for creating programmable optical potentials. DMDs consist of arrays of individually controllable micro-mirrors that can be rapidly reconfigured to create binary intensity patterns, with grayscale levels achieved through temporal modulation or spatial dithering techniques. The Greiner group pioneered the use of DMDs for quantum gas microscopy, demonstrating the ability to project arbitrary potential landscapes onto Mott insulators with microsecond reconfiguration times~\cite{mazurenko_cold-atom_2017}. Their approach enabled dynamic manipulation of lattice geometries and compensation of harmonic confinement to create uniform "box" potentials with improved loading uniformity.

A particularly powerful technique is the "microwave knife" method for site-selective atom removal. This approach combines magnetic field gradients with spatially selective microwave addressing to remove atoms from specific lattice sites with high precision. The technique exploits position-dependent energy shifts for atomic hyperfine states created by magnetic gradients, making each lattice site resonant with a slightly different microwave frequency. Weitenberg et al. first demonstrated this method in $^{87}$Rb quantum gas microscopes, achieving single-site addressing fidelity of approximately 95\%~\cite{weitenberg_single-spin_2011}. The procedure involves applying a magnetic field gradient across the lattice, then using a focused microwave beam to selectively drive transitions at chosen sites, transferring atoms to states that can be subsequently removed using resonant light.

Extensions of the microwave knife technique have enabled the creation of arbitrary patterns of holes in Mott insulators and the realization of custom lattice geometries. Achieved fidelities for pattern creation typically range from 85-95\%, limited primarily by imperfect selectivity of the microwave addressing and residual heating. DMD-based approaches have enabled rapid pattern switching and dynamic manipulation during experiments, with reconfiguration times on microsecond timescales enabling studies of quantum quenches and thermalization dynamics~\cite{choi_exploring_2016}.

% However, these site-selective manipulation techniques share a fundamental limitation: they are primarily "subtractive" methods that can remove atoms from occupied sites but cannot add atoms to empty sites. While they enable precise sculpting of atomic patterns from an initial filled state, achieving arbitrary configurations requires starting with sufficient occupancy everywhere in the pattern. This constraint limits the types of initial states that can be prepared and often requires sacrificing atoms to achieve the desired configuration, reducing the overall system size. Additionally, the fidelity is fundamentally limited by the initial Poissonian loading statistics, as sites that are initially empty cannot be filled using these methods.

% Despite these limitations, SLM and DMD-based manipulation techniques represented a crucial advancement in experimental control over many-body quantum systems, establishing the groundwork for more sophisticated state preparation techniques and demonstrating the importance of precise initial state control for quantum simulation applications.


\textbf{Spin-Selective State Preparation.} Control over the internal spin degrees of freedom of atoms in optical lattices is essential for quantum simulation of magnetic systems and strongly correlated electron models. For $^6$Li, the two lowest hyperfine ground states $|F=\frac{1}{2}, m_F=\pm\frac{1}{2}\rangle$ (commonly denoted $|\uparrow\rangle$ and $|\downarrow\rangle$) provide a natural spin-1/2 system.

Global optical pumping represents the most fundamental approach, transferring populations to a desired hyperfine state with fidelities exceeding 99\%. However, this technique lacks spatial selectivity, preparing uniform spin states across the entire sample. Radiofrequency and microwave transitions enable coherent manipulation of spin states after loading, with spatial selectivity achieved by combining RF pulses with magnetic field gradients~\cite{weitenberg_single-spin_2011}. This approach uses position-dependent Zeeman shifts to make each lattice site resonant with different RF frequencies, enabling single-site spin flips with approximately 95\% fidelity.

Spin-selective optical potentials provide another route by creating different AC Stark shifts for the two spin states, allowing spatial separation or selective manipulation of spin components. The widely adopted "blast" technique exploits distinct optical transitions of different hyperfine states to selectively remove atoms of one spin while leaving the other unaffected. For $^6$Li in magnetic fields, resonant light tuned to one transition ejects atoms in that state while the other remains trapped due to large detuning. Achieved fidelities for selective removal typically exceed 98\%.

Site-resolved spin preparation combining magnetic field gradients with focused optical beams enables preparation of complex spin patterns with single-site resolution. Demonstrations include preparation of antiferromagnetic order in 2D Fermi-Hubbard systems~\cite{mazurenko_cold-atom_2017} and controlled doublon dynamics~\cite{covey_doublon_2016}, achieving stable pair states with high precision.

However, these lattice-based techniques face fundamental limitations compared to constructive approaches. They operate on pre-existing atomic distributions, limiting achievable configurations to modifications of initial Poissonian loading. Empty sites cannot be filled selectively, and achieving specific atom number and spin combinations requires statistical averaging. Additionally, many methods involve heating mechanisms that reduce system coherence, and spin-selective removal necessarily reduces total atom number.

Despite these limitations, lattice-based spin preparation techniques enabled the first demonstrations of engineered magnetic states in ultracold atom systems and established foundations for understanding many-body spin dynamics in optical lattices.


\textbf{Tweezer-Based Loading.} Optical tweezers represent a paradigm shift from stochastic to deterministic loading methods, enabling precise "bottom-up" assembly of many-body quantum systems with arbitrary initial configurations. Unlike traditional loading approaches that rely on statistical distributions, tweezer-based methods provide constructive control over both spatial and internal degrees of freedom, circumventing the fundamental limitations of Poissonian statistics.

The core principle involves using tightly focused laser beams to create microscopic dipole traps that can capture and manipulate individual atoms. The first deterministic preparation of few-fermion systems using optical tweezers with $^6$Li achieved ground-state preparation of 1 to 10 particles with fidelities of approximately 90\%~\cite{serwane_deterministic_2011}. The technique relies on careful control of trap depth and interactions, combined with a "spilling" method that enables exact atom number selection.

The spilling technique exploits the quantized energy levels of fermions in the tight confinement of optical tweezers. By adiabatically reducing the trap depth, atoms occupying the highest energy levels become unbound and escape, while those in lower levels remain trapped. Due to Pauli exclusion, each quantum level accommodates at most one atom per spin state, allowing precise control over the final atom number by selecting the appropriate trap depth. Single-atom loading fidelities exceeding 99\% have been demonstrated in individual tweezers for various atomic systems.

A particularly powerful extension for $^6$Li involves spin-selective spilling using magnetic field gradients. At specific magnetic field strengths, the two hyperfine ground states $|\uparrow\rangle$ and $|\downarrow\rangle$ exhibit different magnetic moments, causing only one spin state to experience significant forces from applied magnetic gradients. This differential response enables selective removal of one spin component while leaving the other unaffected. By carefully tuning the magnetic field and applying controlled gradients during spilling, experimenters can deterministically prepare tweezers containing specific configurations: empty sites, single $|\uparrow\rangle$ atoms, single $|\downarrow\rangle$ atoms, or spin-paired doublons.

The constructive nature of tweezer-based loading provides decisive advantages over subtractive lattice manipulation techniques. Arbitrary spatial patterns can be assembled without requiring initial occupancy at every target site, and specific atom numbers can be placed at chosen locations regardless of initial distribution. This enables preparation of complex initial states that would be statistically unlikely through thermal loading, such as perfect antiferromagnetic order or exotic spin patterns.

Scalability has improved dramatically through advances in optical control systems. Modern implementations using spatial light modulators and digital micromirror devices can generate and control hundreds to thousands of individual tweezers simultaneously~\cite{stuart_single-atom_2018}. Recent demonstrations have shown continuous operation of large-scale atom arrays with more than 1000 atoms~\cite{gyger_continuous_2024}, indicating potential for scaling to system sizes relevant for quantum simulation applications.

The combination of deterministic number control with spin selectivity enables preparation of initial states crucial for studying strongly correlated quantum systems. Controlled few-fermion preparation has been used to realize antiferromagnetic Heisenberg spin chains~\cite{murmann_antiferromagnetic_2015}, demonstrating how precise initial state control enables access to specific many-body quantum phases. The ability to prepare exact Fock states with predetermined spin configurations opens possibilities for studying quantum phase transitions and exotic quantum phases requiring specific initial conditions.

The integration of tweezer arrays with optical lattices represents a powerful hybrid approach. Atoms can be initially assembled in a defect-free tweezer array with the desired spatial and spin configuration, then adiabatically transferred into an optical lattice for many-body dynamics studies. This combination preserves deterministic initial state preparation while providing the periodic potential structure necessary for implementing specific model Hamiltonians. Advanced control capabilities demonstrated in Rydberg tweezer arrays~\cite{bornet_enhancing_2024} show the potential for even more sophisticated manipulation protocols applicable to fermionic systems.

The advantages of tweezer-based loading extend beyond simple state preparation to enable entirely new classes of quantum simulation experiments. The ability to prepare systems with exact atom numbers and spin configurations removes statistical uncertainties that have limited previous studies, enabling precise benchmarking of theoretical predictions and access to quantum phases that require specific particle numbers or spin arrangements.