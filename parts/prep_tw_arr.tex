% !TEX root = ../master-thesis.tex


% \textbf{Tweezer loading.} 
We begin by preparing a spin-balanced mixture in the two lowest hyperfine states, $|1\rangle$ and $|2\rangle$, using a compressed magneto-optical trap (MOT). The atoms are initially loaded into a crossed ODT, where we perform evaporative cooling. This follows closely the sequence described in~\cite{culemann_construction_2024}.

After cooling, the atoms are transferred into a tightly focused optical tweezer potential. The loading process relies on the so-called \emph{dimple trick}~\cite{zurn_few-fermion_2012}, where a tightly confined but deep tweezer potential is superimposed onto the wider ODT reservoir. Because the tweezer affects only a small region of the total cloud, the global temperature $T$ remains approximately unchanged, while the local chemical potential is enhanced. In this regime, the average occupation number $\bar{n}(E_i)$ of a single-particle state $i$ with energy $E_i$ follows the Fermi-Dirac distribution:
\begin{equation}
    \bar{n}(E_i) = \frac{1}{e^{(E_i - \mu)/k_B T} + 1}.
\end{equation}
If the energy gap between the ground state $E_0$ and the Fermi energy $E_F$ is increased such that $(E_0 - E_F) / k_B \gg T$, then $\bar{n}(E_0) \rightarrow 1$. This ensures near-unity occupation of the lowest level, which provides an ideal starting point for deterministic preparation. In our experiment, this condition is achieved by ramping on the tweezer adiabatically while continuing evaporation inside the tweezer. The full loading and cooling sequence is depicted in Fig.~\ref{fig:preparationseq}.

% \textbf{Deterministic few-body preparation via spilling.} 
To isolate a well-defined number of atoms in the lowest motional states of the tweezer, we use the \emph{spilling technique}, as described in~\cite{zurn_few-fermion_2012, holten_pauli_2022}. This method relies on tilting the potential with a magnetic field gradient and reducing the trap depth to allow atoms above a threshold energy to tunnel out. 
The resulting states are shown in Fig.~\ref{fig:preparation}.
% The energy levels and wavefunctions of the effective 1D potential under combined optical and magnetic fields were obtained numerically using a finite-difference method and the Thomas algorithm.

The spilling sequence is performed at a magnetic field of 527\,G, where the two spin states are nearly non-interacting. A magnetic field gradient of 20\,G/cm creates a linear tilt, and the tweezer power is lowered to a value that sets the spill threshold. After a short tunneling time, the trap depth is ramped back up to recapture the remaining atoms. By empirically optimizing these parameters, we achieve deterministic preparation of two atoms per tweezer with a fidelity of 90\,\%. This sequence results in high-fidelity preparation within a total experimental cycle time of less than 2\,s.



\begin{figure}
    \centering
    \includegraphics{fig-py/step-plot-2d.pdf}
    \caption[2D step plot]{
        \textbf{2D step plot.}
        Measured 2D step plot as a function of tweezer power and magnetic field gradient. Each point indicates the average atom number obtained for a given combination of parameters. This map confirms that for any spill power, a suitable magnetic gradient can be found to achieve a desired quantized atom number.
    }
    \label{fig:spillingadd-2d}
\end{figure}

