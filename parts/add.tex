% !TEX root = ../master-thesis.tex

\textbf{From frequency to position}. И в camera based balancing, и в atoms based balancing для обработки изображений удобно определить афинное преобразование из frequency space to position space:
\begin{equation*}
	\vc{r} = H \vc{\omega}
	\hspace{5 mm} \leftrightarrow \hspace{5 mm} 
	\begin{pmatrix}
		x \\ y
	\end{pmatrix} = \begin{pmatrix}
		h_{11} & h_{12} & h_{13} \\
		h_{21} & h_{22} & h_{23} \\
	\end{pmatrix} 
	\begin{pmatrix}
		\sub{\omega}{hor} \\
		\sub{\omega}{ver} \\
		1
	\end{pmatrix}.
\end{equation*}
Можно, например, для случайных $\vc{\omega}_j \in [\sub{\omega}{min},\, \sub{\omega}{max}]$ измерить $\vc{r}_j$, таким образом сформировав две матрицы $\omega_{ij}$ with $i \in \{\mathrm{hor},\, \mathrm{ver}\}$ and $r_{ij}$ with $i \in \{x, y\}$. Остается решить уравнение на $H$ (что соответсвует Least squares method):
\begin{equation}
	r = H \omega,
	\hspace{0.5cm} \Rightarrow \hspace{0.5cm}
	r \omega\T = H \omega \omega\T
	\hspace{0.5cm} \Rightarrow \hspace{0.5cm}
	r \omega\T \left(\omega \omega\T\right)^{-1} = H.
	\label{LinReg:freq2pos}
\end{equation}