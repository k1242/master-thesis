% !TEX root = ../master-thesis.tex

% \textbf{Cold atoms advantages.} 
Ultracold atomic gases provide an exceptionally clean and tunable platform for realizing the Fermi-Hubbard model with unprecedented control over system parameters \cite{esslinger_fermi-hubbard_2010,gross_quantum_2017}. Neutral fermionic atoms such as $^6$Li trapped in optical lattices created by interfering laser beams naturally implement the kinetic energy term through tunneling between neighboring sites. The interaction strength $U$ can be continuously tuned using Feshbach resonances, where an external magnetic field controls the scattering length between atoms in different hyperfine spin states. This allows experimental exploration of the entire phase diagram from the non-interacting limit to strongly correlated regimes. Site-dependent potentials can be introduced using digital micromirror devices or spatial light modulators, enabling controlled studies of disorder effects and many-body localization \cite{choi_exploring_2016,schreiber_observation_2015}.

The development of quantum gas microscopy has revolutionized the field by providing single-site resolution imaging of atomic lattice systems \cite{bakr_quantum_2009,sherson_single-atom-resolved_2010}. These techniques enable direct measurement of local observables such as density correlations, magnetic structure factors, and spin-resolved occupations that are inaccessible in solid-state systems \cite{gross_quantum_2021}. Recent achievements include the observation of antiferromagnetic correlations in two-dimensional Fermi-Hubbard systems \cite{mazurenko_cold-atom_2017,parsons_site-resolved_2016} and demonstrations of many-body localization in disordered lattices \cite{bordia_probing_2017}.

% \textbf{Current limitations.} 
Despite these advances, conventional optical lattice experiments face fundamental constraints in state preparation that limit their potential for studying complex many-body phenomena. Thermal loading from magneto-optical traps produces statistical filling with Poissonian atom number fluctuations, resulting in random defects and uncontrolled entropy that obscure the underlying physics \cite{esslinger_fermi-hubbard_2010}. The harmonic confinement typically present in these systems creates spatial inhomogeneity through varying local chemical potential, leading to "wedding cake" structures with different filling factors across the trap. Additionally, achieving specific initial states required for studying dynamical phases or controlled doping profiles remains challenging with conventional loading methods.

% \textbf{Tweezer innovation.} 
Optical tweezer arrays offer a transformative solution to these state preparation challenges by enabling deterministic, bottom-up assembly of many-body quantum systems \cite{browaeys_many-body_2020}. Individual atoms can be captured and manipulated in tightly focused laser beams, providing single-site control over both spatial and internal degrees of freedom. Recent demonstrations have shown the feasibility of implementing few-body Fermi-Hubbard dynamics in programmable tweezer geometries \cite{spar_realization_2022,yan_two-dimensional_2022}, establishing the foundation for scalable fermionic quantum simulators.

This work advances tweezer-based approaches through the development of crossed acousto-optic deflector (AOD) systems that enable rapid reconfiguration of two-dimensional arrays. The key innovation lies in implementing spin-selective manipulation protocols that allow preparation of arbitrary site- and spin-resolved occupation patterns. Combined with novel imaging techniques for spin-resolved single-shot detection, this platform provides the experimental tools necessary for systematic studies of both equilibrium and dynamical Fermi-Hubbard physics.

% \textbf{Separation of preparation and dynamics.} 
The tweezer-based approach exploits a fundamental insight: optimal state preparation and Hamiltonian evolution can be achieved using different experimental configurations. High-fidelity initial state preparation is performed in the tweezer array, where strong confinement and individual site control enable deterministic loading and spin manipulation. Subsequently, atoms are transferred to an optical lattice optimized for implementing the desired Hamiltonian evolution with precise control over tunneling and interaction parameters. This separation enables access to low-entropy initial states that would be statistically unlikely under thermal loading, opening new possibilities for studying quantum phase transitions, out-of-equilibrium dynamics, and exotic correlated phases.