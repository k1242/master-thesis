% !TEX root = ../master-thesis.tex

% --------------------------------------------------------------------------------------
% Intro
% --------------------------------------------------------------------------------------

% \textbf{Overview.}
Understanding the dynamics of isolated quantum systems remains one of the central goals of contemporary many-body physics. Over the last decades, considerable progress has been made in classifying and probing different dynamical regimes, from thermalizing phases consistent with conventional statistical mechanics to exotic non-ergodic phases that violate the Eigenstate Thermalization Hypothesis (ETH). Among these, the Fermi-Hubbard model has emerged as a paradigmatic platform for studying the interplay of interactions, quantum statistics, and disorder in strongly correlated systems.

In the clean, disorder-free limit, the Fermi-Hubbard model exhibits rich equilibrium physics, including Mott insulators, spin ordering, and pseudogap phenomena relevant to high-temperature superconductivity \cite{esslinger_fermi-hubbard_2010}. However, the model also serves as a fertile ground for exploring nonequilibrium phenomena, such as quantum quenches, relaxation, transport, and entanglement dynamics — especially when generalized to include disorder or spatial inhomogeneities.

Recently, theoretical and experimental attention has increasingly shifted toward the role of disorder in quantum many-body dynamics. It is now understood that disorder can lead to fundamentally different behaviors depending on the presence or absence of interactions. For instance:
\begin{itemize}
	\item In the absence of interactions, disorder induces \textit{Anderson localization}, which prevents particle diffusion and leads to persistent memory of initial conditions \cite{anderson_absence_1958}.
	\item When interactions are present, the system may enter the regime of \textit{many-body localization (MBL)}, characterized by the absence of thermalization and slow unbounded entanglement growth \cite{basko_metalinsulator_2006,nandkishore_many-body_2015}.
	% , and emergent local integrals of motion
	\item In contrast, when disorder is weak or absent, the system typically evolves toward local thermal equilibrium, consistent with the predictions of \textit{ETH} \cite{deutsch_quantum_1991,srednicki_chaos_1994}. 
\end{itemize}

These dynamical phases (\emph{thermal, Anderson-localized, and MBL}) are typically distinguished through the behavior of local observables, spectral statistics, and the dynamics of quantum correlations. Their interplay is especially rich in two dimensions \cite{abanin_colloquium_2019}.

From an experimental standpoint, studying such phenomena requires precise control over initial states, evolution Hamiltonians, and high-fidelity measurements of observables at the single-site level. This thesis presents the development of experimental capabilities toward such control, demonstrating deterministic initialization of fermionic states in a two-dimensional tweezer array and spin- and site-resolved imaging, which together provide the necessary tools for future studies of programmable Fermi-Hubbard dynamics.

Compared to conventional optical lattice experiments, the tweezer-based approach offers several key advantages for nonequilibrium quantum simulation:
\begin{enumerate}
	\item \textit{Deterministic and programmable state preparation.} Using a sequence of global and spin-selective spilling operations, arbitrary configurations of fermionic atoms can be prepared with high fidelity. This capability enables initialization of tailored many-body states for probing specific dynamical scenarios, such as local quenches, domain-wall melting, or imbalance relaxation.
	\item \textit{Fast experimental cycle and large statistics.} The entire experimental sequence, including preparation, evolution, and measurement, completes in under two seconds, allowing up to $10^5$ experimental repetitions per day. This rapid repetition rate is crucial for averaging over disorder realizations and collecting sufficient statistics for dynamical observables.
	\item \textit{Spin-resolved detection capabilities.} The developed imaging system demonstrates fluorescence-based discrimination of atomic spin states, providing the technical foundation necessary for future single-shot, site-resolved measurements of observables such as density profiles $\langle n_j \rangle$, magnetization $\langle \sigma^z_j \rangle$, and spin correlations $\langle \sigma^z_i \sigma^z_j \rangle$.
\end{enumerate}

In tandem with experimental capabilities, in this work a numerical simulation package was developed for modeling real-time dynamics in finite-size Hubbard systems. The package combines exact diagonalization (ED) for small systems and Krylov subspace methods for larger Hilbert spaces (up to $10^9$ dimensions), and supports evaluation of observables and entanglement entropy in arbitrary geometries and disorder realizations.

Together, these tools enable a systematic exploration of the dynamical phase diagram of the disordered Fermi-Hubbard model in two dimensions. By leveraging control over initial conditions, disorder strength, and interactions, as well as the ability to access key observables, one can address central questions in nonequilibrium many-body physics: What determines whether a system thermalizes? When does localization persist in the presence of interactions? How do correlations and entanglement spread in different regimes?

In the following subsections, we review the theoretical framework underpinning these questions, beginning with the concept of thermalization in isolated quantum systems.

