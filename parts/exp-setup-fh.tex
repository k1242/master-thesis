% !TEX root = ../master-thesis.tex

The Fermi-Hubbard model is a cornerstone theoretical framework for describing strongly correlated electron systems, particularly relevant in condensed matter physics for understanding the behavior of high-temperature superconductors, such as cuprates \cite{koepsell_quantum_2021}. The Hamiltonian of the 2D Fermi-Hubbard model is typically expressed as:
\begin{equation}
H = -t \sum_{\langle i,j \rangle, \sigma} \left( c_{i,\sigma}^\dagger c_{j,\sigma} + \text{h.c.} \right) + U \sum_i n_{i,\uparrow} n_{i,\downarrow} + \sum_{i,\sigma} \varepsilon_i n_{i,\sigma},
\label{eq:fermi-hubbard}
\end{equation}
where the first term represents hopping with amplitude $t$ between nearest-neighbor lattice sites, the second term describes the on-site interaction energy $U$, and the third term introduces site-dependent energy offsets $\varepsilon_i$, accounting for disorder or external potentials \cite{koepsell_quantum_2021}.

In physical terms, the competition between kinetic energy, captured by the hopping term, and potential energy, represented by the on-site interaction, gives rise to rich emergent phenomena. At half-filling and sufficiently large interaction strength $U \gg t$, the model predicts the formation of a Mott insulating phase, characterized by suppressed conductivity due to electron localization. At lower temperatures, antiferromagnetic correlations dominate, leading to spin ordering. Upon doping, the system can exhibit pseudogap behavior and potentially unconventional superconductivity analogous to that observed in cuprates, although achieving clear signatures of superconductivity in numerical and experimental studies remains challenging \cite{koepsell_quantum_2021}.

Beyond the single-layer scenario, the bilayer 2D Fermi-Hubbard model offers additional intriguing phenomena, including enhanced pairing mechanisms and novel magnetic orders. Experimentally realizing such bilayer systems could provide critical insights into mechanisms underlying high-temperature superconductivity. Within the UniRand experiment, employing an accordion lattice geometry makes the exploration of bilayer configurations feasible, thereby opening new avenues for investigating these phenomena \cite{huang_construction_2024}.

In the context of ultracold atoms, the Fermi-Hubbard model can be realized by trapping fermionic atoms, such as $^6$Li, in optical lattices, where parameters of the Hamiltonian can be precisely tuned. Specifically, the interaction parameter $U$ is controlled via Feshbach resonances, where an external magnetic field is adjusted to tune the scattering length between atoms in different hyperfine states. This allows continuous tuning from attractive to strongly repulsive interactions, enabling experimental exploration of the full phase diagram \cite{culemann_construction_2024}.

The hopping amplitude $t$ is controlled by adjusting the depth of the optical lattice potential. Deeper lattices decrease the tunneling rate, effectively increasing the ratio $U/t$ and stabilizing strongly correlated insulating phases. Additionally, site-dependent potentials $\varepsilon_i$ can be introduced via digital micromirror devices (DMD), allowing the controlled introduction of disorder or custom potential landscapes essential for exploring Anderson or many-body localization phenomena.

State preparation is envisioned to proceed via an initial deterministic loading of atoms into optical tweezer arrays, followed by a transfer into optical lattices. Such a procedure offers precise control over initial conditions, vital for exploring complex dynamical processes. Furthermore, measurement protocols incorporating Random Unitary Protocols, involving sequences of local random quenches and subsequent measurements, can yield critical information about the entanglement entropy and many-body coherence, significantly enriching experimental observables \cite{culemann_construction_2024, huang_construction_2024}.

% The experimental setup employed by UniRand, detailed further in subsection \ref{subsec:exp-setup-overview}, follows a standard ultracold atom preparation sequence: atoms emitted from an oven are initially captured and cooled in a two-dimensional magneto-optical trap (2D MOT), then transferred into a three-dimensional MOT (3D MOT). Subsequently, atoms are loaded into an optical dipole trap (ODT), further cooled, and finally transferred into an optical tweezer array for deterministic state preparation. The future integration of optical lattices will complete the experimental realization of the Fermi-Hubbard model and its diverse dynamical phases.

