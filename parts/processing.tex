На каждой из фотографий в каждой выделенной области хочется уметь отличать наличие атома от его отсутствия, что в контексте наличия шумов становится нетривиальным. Можно было бы просто посчитать Pixel Integral \red{(добавить рисунок, а)}, но можно лучше. Расположим оптику (схема оптики) таким образом, что в среднем на пиксель приходилось по фотону. \red{Измерим} какой сигнал на один фотон мы ожидаем и в соответсвии с этим выставим threshold. Таким образом может быть отфильтрована большая часть шума \red{(b)}. Но дальше можно воспользоваться информацией о том, что атомы излучают кучно \red{(пример двух фото с одним counts, с атомом и без)}, в отличие от случайного шума. Так что отфильтровав низкие частоты, свернув изображение с гауссовым фильтром, пролучаем \red{(c)}. 














