% !TEX root = ../master-thesis.tex

\textbf{Technical innovations developed.} This work addresses the key experimental bottlenecks limiting progress in Fermi-Hubbard quantum simulation through the development of three interconnected technological advances. The first innovation is a spin-resolved free-space imaging system for $^6$Li atoms that enables simultaneous detection of both hyperfine ground states in a single experimental shot. Unlike previous sequential imaging approaches \cite{bergschneider_spin-resolved_2018}, the implementation utilizes stretched states $|3\rangle$ and $|6\rangle$ with polarization-selective detection paths, achieving \red{99\%} spin discrimination fidelity while maintaining fast imaging times of 20 $\mu$s. This capability provides direct access to local magnetization and spin correlations essential for characterizing magnetic phases and dynamical regimes.

The second major contribution involves the design and control of two-dimensional optical tweezer arrays using crossed acousto-optic deflectors. This work develops calibration protocols that achieve uniform tweezer depths across arrays. A key innovation is the implementation of spin-selective spilling techniques that exploit differential magnetic moments of hyperfine states to enable deterministic preparation of arbitrary site- and spin-resolved occupation patterns. This approach represents the first demonstration of programmable spin-selective state preparation using crossed AOD configurations.

The third component consists of numerical simulation tools that provide theoretical benchmarks for experimental protocols. A custom GPU-accelerated software package was developed that combines exact diagonalization with Krylov subspace methods, enabling simulation of Fermi-Hubbard dynamics in systems up to $10^9$ dimensions. The package supports arbitrary geometries while efficiently computing observables including local densities, spin correlations, and entanglement entropy.

\textbf{Numerical roadmap for dynamical phases.} The computational framework developed in this work serves as both a validation tool for experimental protocols and a roadmap for near-term studies of quantum thermalization and localization. Systematic simulations demonstrate clear signatures that distinguish thermal, Anderson localized, and many-body localized phases through experimentally accessible observables such as imbalance relaxation and entanglement growth. These results provide concrete targets for upcoming experiments and establish the parameter regimes where different dynamical behaviors can be reliably observed with the developed experimental tools.

\textbf{Thesis statement.} This work develops key experimental tools for studying both equilibrium and dynamical aspects of the Fermi-Hubbard model in ultracold atomic systems. By combining deterministic state preparation in programmable tweezer arrays with spin-resolved imaging, the platform enables systematic investigation of magnetic correlations relevant to quantum thermalization phenomena in strongly correlated fermions.

\textbf{Chapter roadmap.} The experimental platform and its capabilities are presented through a sequence of technical developments that build toward the physics applications. Section 2 provides an overview of the overall experimental apparatus. Section 3 details the implementation of spin-resolved free-space imaging, including the optical setup, image processing algorithms, and characterization of detection fidelity. Section 4 describes the tweezer array system, covering AOD control, calibration procedures, and spin-selective state preparation protocols. Section 5 presents the numerical simulation framework and demonstrates its application to studies of dynamical phases in small Fermi-Hubbard systems. Together, these tools establish a foundation for future studies of strongly correlated quantum matter using fermionic systems.