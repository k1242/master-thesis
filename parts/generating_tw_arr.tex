
\textbf{Acousto Optic Deflector (AOD)}. AOD, как и AOM, состоит из кристалла, который модулируется пьезоэлементом. Проходящие через кристалл фотоны $(\sub{\vc{k}}{in}, \sub{\omega}{in})$ рассеиваются на фононах $(\vc{q}, \Omega)$ via Bragg diffraction. To have higher efficiency we need to satisfy Bragg condition (проверить и добавить источник)
\begin{equation*}
	\sub{n}{sc} q = \sub{k}{in} \sin(\theta),
\end{equation*}
\grey{где $\theta$ это угол между $\vc{k}$ и нормалью к $\vc{q}$} \red{(добавить рисунок)}. Внутри AOD находится несколько пьезоэлементов, к которым ведут провода подобранной длины так, чтобы при изменение частоты $\Omega$ направление $\vc{q}$ менялось соответсвующим Bragg condition образом. Это помогает улучшить диффракционную эффективность \grey{(добавить определение или ссылку)} AOD. На выходе полуются $(\sub{\vc{k}}{out}, \sub{\omega}{out}) = (\sub{\vc{k}}{in}+\vc{q}, \sub{\omega}{in} + \Omega)$. 
Имея набор частот в модулирующем сигнале $(\vc{q}_j, \Omega_j)$ получим на выходе набор лучей
\begin{equation*}
	(p_j, \vc{k}_j, \omega_j) = (F_j(\vc{a}, \sub{\vc{\omega}}{in}), \sub{\vc{k}}{in}+\vc{q}_j, \sub{\omega}{in} + \Omega_j),
\end{equation*}
c мощностью в каждом луче на выходе $p_j$. Регулируя вектор амплитуд $\vc{a}$, подающихся в AOD можно контролировать выходную мощность $\vc{p}$. 
