

% \textbf{Acousto Optic Deflector (AOD)}. AOD, как и AOM, состоит из кристалла, который модулируется пьезоэлементом. Проходящие через кристалл фотоны $(\sub{\vc{k}}{in}, \sub{\omega}{in})$ рассеиваются на фононах $(\vc{q}, \Omega)$ via Bragg diffraction. To have higher efficiency we need to satisfy Bragg condition (проверить и добавить источник)
% \begin{equation*}
% 	\sub{n}{sc} q = \sub{k}{in} \sin(\theta),
% \end{equation*}
% \grey{где $\theta$ это угол между $\vc{k}$ и нормалью к $\vc{q}$} \red{(добавить рисунок)}. Внутри AOD находится несколько пьезоэлементов, к которым ведут провода подобранной длины так, чтобы при изменение частоты $\Omega$ направление $\vc{q}$ менялось соответсвующим Bragg condition образом. Это помогает улучшить диффракционную эффективность \grey{(добавить определение или ссылку)} AOD. На выходе полуются $(\sub{\vc{k}}{out}, \sub{\omega}{out}) = (\sub{\vc{k}}{in}+\vc{q}, \sub{\omega}{in} + \Omega)$. 
% Имея набор частот в модулирующем сигнале $(\vc{q}_j, \Omega_j)$ получим на выходе набор лучей
% \begin{equation*}
% 	(p_j, \vc{k}_j, \omega_j) = (F_j(\vc{a}, \sub{\vc{\omega}}{in}), \sub{\vc{k}}{in}+\vc{q}_j, \sub{\omega}{in} + \Omega_j),
% \end{equation*}
% c мощностью в каждом луче на выходе $p_j$. Регулируя вектор амплитуд $\vc{a}$, подающихся в AOD можно контролировать выходную мощность $\vc{p}$. 



\textbf{Beam collimation and polarization.} The tweezer array is formed by delivering light from a fiber outcoupler and collimating it with an $f = 40\,\mathrm{mm}$ achromatic lens mounted on a translation stage for precise control. To ensure efficient diffraction through the acousto-optic deflectors (AODs), horizontal polarization is set using a $\lambda/2$ waveplate and a polarizing beam splitter (PBS). Correct alignment of the PBS is verified by tracking the beam position on a camera before and after insertion.

\textbf{Acousto-optic deflectors and relay imaging.} The array is generated using a pair of orthogonally mounted AODs, each mounted on custom blocks to maintain a common beam height of 100\,mm. The beam is guided into the first AOD using two mirrors, and its alignment is optimized to maximize both transmission and diffraction efficiency (typically exceeding 90\% at 65\,MHz). The two AODs are connected via a 4f relay built from achromatic lenses in a 30\,mm cage system. Precise positioning is achieved by aligning the relay externally using collimated light and checking for minimal wavefront distortion on a shear plate. An iris at the Fourier plane of the relay filters out the zeroth diffraction order.

\textbf{Telescope and beam expansion.} After the second AOD, the beam is expanded using a telescope consisting of $f = 150\,\mathrm{mm}$ and $f = 500\,\mathrm{mm}$ lenses. The alignment ensures that the beam is collimated and centered on both lenses. The position of the second lens is mechanically fixed, while the telescope length is adjusted via mirror positions to achieve good collimation, verified with a shear plate. The zeroth-order beam after the second AOD is blocked at the intermediate focus.

\textbf{Monitoring and power balancing.} A flip mirror is installed in the beam path to optionally redirect the light into a monitoring camera without disturbing the main optical alignment. This enables fast access to the tweezer array profile during alignment or balancing procedures.

We avoid using a back-side polished mirror for beam sampling in front of the camera. Although commonly employed for its simplicity, such mirrors introduce (\red{? add measured data: october 2024}) spatially varying interference fringes due to reflections from different internal surfaces of the substrate. These fringes distort the measured intensity distribution, especially for rays entering the camera at different angles and positions. This effect becomes critical when calibrating the response of the AODs, as it leads to systematic errors in measured beam uniformity. Instead, we sample the beam with a removable flip mirror that fully redirects the beam, ensuring an undistorted and angle-independent intensity profile at the camera plane.
