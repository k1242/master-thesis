% !TEX root = ../master-thesis.tex

% \textbf{High-temperature superconductivity.} 
The discovery of unconventional superconductivity in copper-oxide ceramics with critical temperatures reaching 135 K represents one of the most significant puzzles in condensed matter physics. These materials exhibit a complex phase diagram featuring strange metallic behavior that violates conventional Fermi liquid theory, pseudogap phenomena emerging from antiferromagnetic Mott insulating parent compounds, and the eventual emergence of d-wave superconductivity upon doping \cite{koepsell_quantum_2021}. Theoretical progress has been hampered by the strongly correlated nature of electrons in these systems, where simple perturbative approaches fail and the interplay between magnetic interactions, charge dynamics, and pairing mechanisms remains poorly understood.

The Fermi-Hubbard model, defined by the Hamiltonian
\begin{equation}
H = -t \sum_{\langle i,j \rangle, \sigma} \left( c_{i,\sigma}^\dagger c_{j,\sigma} + \text{h.c.} \right) + U \sum_i n_{i,\uparrow} n_{i,\downarrow} + \sum_{i,\sigma} \varepsilon_i n_{i,\sigma},
\end{equation}
captures the essential physics believed to underlie high-temperature superconductivity. The competition between kinetic energy (hopping amplitude $t$) and on-site Coulomb repulsion ($U$) gives rise to the rich phase diagram observed in cuprates, while site-dependent potentials $\varepsilon_i$ allow controlled introduction of disorder.

% \textbf{Quantum thermalization and localization.} 
Beyond equilibrium phenomena, the Fermi-Hubbard model serves as a paradigmatic system for understanding fundamental questions about quantum statistical mechanics in isolated many-body systems. The Eigenstate Thermalization Hypothesis (ETH) predicts that generic quantum systems evolve toward thermal equilibrium, with local observables losing memory of initial conditions \cite{deutsch_quantum_1991,srednicki_chaos_1994}. However, the presence of disorder can dramatically alter this behavior, leading to Anderson localization in non-interacting systems where all single-particle states become exponentially localized \cite{anderson_absence_1958,billy_direct_2008,roati_anderson_2008}. The interplay between interactions and disorder gives rise to many-body localization (MBL), a non-thermal phase where strongly interacting systems fail to thermalize despite high energy density \cite{basko_metalinsulator_2006,nandkishore_many-body_2015,abanin_colloquium_2019}. MBL systems exhibit characteristic logarithmic growth of entanglement entropy, violation of ETH, and persistent memory of initial conditions. Recent experimental observations of MBL signatures in both one- and two-dimensional ultracold atom systems have opened new avenues for studying these exotic quantum phases \cite{schreiber_observation_2015,choi_exploring_2016,bordia_probing_2017}.

% \textbf{Competition of energy scales.} 
% The rich physics of the Fermi-Hubbard model emerges from the competition between three characteristic energy scales: the kinetic energy set by the hopping amplitude $t$, the interaction energy characterized by the on-site repulsion $U$, and the disorder strength parameterized by the variance of $\varepsilon_i$. 
In the strongly interacting regime $U \gg t$, the system exhibits Mott insulating behavior with suppressed charge fluctuations and emergent magnetic ordering \cite{mazurenko_cold-atom_2017}. Weak disorder can enhance localization effects, while strong disorder can drive transitions between thermal and many-body localized phases. Understanding the dynamical phase diagram in the space of these competing energy scales represents one of a central challenges in quantum many-body physics.