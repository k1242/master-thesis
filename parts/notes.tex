% !TEX root = ../master-thesis.tex


\section*{Мысли про текст}

\begin{enumerate}
	\item Добавить мостики к Tweezer movement и Tweezer loading
	\item Higlight arbitrary spin-resolved configurations 4.8
	\item Обсудить fidelity процесса, что на него влияется. С каким fidelity происходит additional spilling
	% \item Storyline
	% \item Каждая картинка должна быть описана и понятна. 
	% \item Описание unirand эксперимента: подробнее про структуру уровней, про картинки.
	% \item Вводный рассказ про Fermi Hubbard
	% \item Tweezer loading
	% \item Tweezer movement
	\item ? добавить в chapter 6 про добавки с finite size к термализации из эргодического бассейна
	% \item ? MWM
	% \item BMF as SAT task
	\item ? Написать про добавки к obj function в линейной модели
	
	% \item Atom based measurements: SVF, , atom-based crosstalk (and comparison)
	
	% \item site- and spin- resolved state preparation

	% \item non-factorizable state preparation, add large Li imgs (see movie)
	 % Можно а-ля the Li добавлять сверху к средним картинкам расшифровку

	% \item Графики для антенн с фитом

\end{enumerate}



\section*{Мысли про figures}


\begin{itemize}
	\item ! Make a sketch:  splitter (PBS) after the objective, which directs $\sigma\pm$ components to different regions of the detector (see Fig. 7).
	\item ! That needs to be justified, there could be couplings between H and V (via the angle of the beams entering the second AOD)
	\item ? Spin-selective spilling (You should also sketch the potential for both spin states. Hard to understand otherwise)
	\item ? Схема стабилизации лазеров, какие отстройки, какие частоты, в контексте imaging
	% \item Single atom counting
	\item Демонстрация с Random Unitaries (Xinyi тезис)
	\item Схема установки, фото 3D mot
	% \item BEC
	\item Imaging: разница двух облачков (один, два continous, два alternating)
	% \item Imaging: histogram noise vs atoms, raw nuvu img
	% \item Flashing. Экспериментальная установка, табличка с её параметрами
	% \item State preparation: spilling. Схематичное изображение (посмотреть в Heidelberg thesis).
	% \item ? Экспериментальная последовательность
	\item ? Feshbach resonance
	% \item ? loading issues
	\item ? MWM (simulation, observed)
	\item ? Theory: описание fermi-hubbard, фазовая диаграмма (посмотреть coepsill)
	\item ? Theory: вклад от лабиринтов в локализацию
	% \item ? 2D Step Plot
\end{itemize}





% Huang: 17-25
% Culemann: 15-24




% Sec.\ref{sec:intro} or subsection\ref{subsec:control} Fig.\ref{fig:stepplot} or \eqref{eq:evolution}