% document's head

\begin{center}
    \LARGE \textsc{Fermionic State Preparation and Imaging \\ in Optical Tweezer Array}
\end{center}

\hrule

\phantom{42}

\begin{flushright}
    \begin{tabular}{rr}
    % written by:
        % \textbf{Источник}: 
        % & \href{__ссылка__}{__название__} \\
        % & \\
        % \textbf{Лектор}: 
        % & _ФИО_ \\
        % & \\
        \textbf{Author}: 
        & Khoruzhii Kirill \\
        % & Примак Евгений \\
        & \\
    % date:
        \textbf{Date}: &
        \textit{\today}\\
    \end{tabular}
\end{flushright}

\thispagestyle{empty}

\tableofcontents

\vfill

Базовая структура диплома:
\begin{enumerate*}
    \item deterministic state preparation (single tweezer)
    \begin{enumerate*}
        \item loading (2D MOT, MOT, Dipol Trap, Tweezer)
        \item spilling
    \end{enumerate*}
    \item single-atom spin resolved free space imaging
    \begin{enumerate*}
        \item ! flashing and model
        \item ! image processing
    \end{enumerate*}
    \item deterministic state preparation (tweezer array)
    \begin{enumerate*}
        \item ! generating
        \item ! control
        \item ! balancing
    \end{enumerate*}
\end{enumerate*}

Это история про то как сделать и сфотографировать спиновое состояние в твизере.     


\newpage
